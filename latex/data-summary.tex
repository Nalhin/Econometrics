\section{Opis danych}\label{sec:opis-danych}

\subsection{Pochodzenie}\label{subsec:pochodzenie}
Dane pochodzą z serwisu Keggle\cite{berlin-airbnb-data}.
Zostały one zebrane na podstawie ofert wystawionych na serwisie Airbnb dnia 7 listopada 2018 roku.
Pozyskiwanie zostały za pomocom technologi do webscrappingu.
Proces ten polega na pobraniu kodu HTML strony internetowej i przeparsowanie odpowiednich znaczników do postaci pliku csv.

\subsection{Czyszczenie danych - Reguła Trzech Sigm}\label{subsec:czyszczenie-danych-reguła-trzech-sigm}

Z uwagi na specyfikacje pozyskania danych (występowanie błedów ludzkich) zdecydowałem sie na wykorzystanie reguły trzech sigm podczas czyszczenia danych.
Reguła trzech sigm dla danego rozkładu normalnego \(N(\mu , \sigma)\) oznacza, że w przedziale \([\mu - 3\sigma ,\mu + 3\sigma ] \) znajduję się 99.7 \% wszystkich obserwacji.
Odstające (okolo 1500) obserwacje zostały usunięte z dalszej analizy.

\subsection{Braki danych}\label{subsec:problem-braku-danych}
W modelu znajdowała się niewielka ilość braków (okolo 50).
Z uwagi na duża liczbę badanych obserwacji, braki zostały wyrzucone z dalszej analizy.

\subsection{Zmienne}\label{subsec:zmienne}

\customplot{CorrMatrix}{Macierz korelacji zmiennych}{corr-matrix}

\subsubsection{Price}\label{subsubsec:price}
Zmienna objaśniana w modelu.
Określa ona cene za wynajem (w dolarach) na okres jednej doby.


\customtable{Price}
\customboxplot{Price}

Wnioski

\begin{itemize}
    \item Wysoka skośnośc oznacza, że ceny są przesunięte w prawo.
    \item Wysoka kurtoza może oznaczać występowanie dodatnich outlierow (widoczne na wykresie), jednak z uwagi na specyfikacje zmiennej (na Airbnb można wynajmować nawet apartamenty), takowe nie bedą usuwane.
    \item Zmienna ma wariacje na wysokim poziomie, może to wynikać z dużych róznic ceny mieszkań zależnie od ich wielkości (i ilości osób mogących się w takowym zmieścić).
\end{itemize}

\subsubsection{Number of Reviews}\label{subsubsec:number-of-reviews}
Zmienna objaśniająca w modelu.
Określa ilość ocen pod ogłoszeniem. Oznacza to wysoką popularnościa danego miejsca jak również wysokie standardy, gdyż ludzie wystawiają częsciej przychylne oferty niż negatywne.

\customtable{NumberOfReviews}

\customboxplot{NumberOfReviews}

Wnioski

\begin{itemize}
    \item Większość ofert ma niską ilośc opini, jednak rekordziści osiągają ich ponad 120.
    \item Oferty bez opini sa nieliczne.
    Należy pamiętać, że na tego typu serwisie można w sztuczny sposób zwiększyć ilość opinii (np prosząc znajomych).
\end{itemize}

\subsubsection{Minimum nights}\label{subsubsec:minimum-nights}
Zmienna objaśniająca w modelu.
Opisuje na jaki jest minimalny czas wynajmu danej nieruchomości.

\customtable{MinimumNights}

\customboxplot{MinimumNights}

Wnioski

\begin{itemize}
    \item Większość ofert wymaga wynajmu na minimum 3 dni.
    Wiąże się to z kosztami jakie ponosi wynajmujący w sytuacji zmiany wynajmującego (np sprzątanie).
    \item Skrajne obserwacje to najprawdopodobniej oferty normalnego wynajmu (nie krotkoterminowego) szukających rozgłosu poprzez umieszczenie ich na Airbnb.
\end{itemize}

\subsubsection{Bathrooms}\label{subsubsec:bathrooms}
Zmienna objaśniająca w modelu.
Określa ilość toalet w danej nieruchomości.
Jest to zmienna typu skokowego.

\customtable{Bathrooms}

\customboxplot{Bathrooms}

Wnioski

\begin{itemize}
    \item Kurtoza osiąga olbrzymią wartość.
    Oznacza to ze znacza część ofert ma własnie jedna toalete.
    W większości mieszkań wprowadzenie dodatkowej toalety wiązałoby się ze zmniejszeniem ilości pokoi, a co za tym idzie zysków.
    \item Oferty ze skrajnie duża ilościa toalet pochodza najprawdopodobniej z hoteli.
\end{itemize}

\subsubsection{Bedrooms}\label{subsubsec:bedrooms}
Zmienna objaśniająca w modelu.
Opisuje ile oddzielnych sypialni zostanie powierzone wynajmującemu.
Należy zauważyc, że jest to zmienna typu skokowego.

\customtable{Bedrooms}

\customboxplot{Bedrooms}

Wnioski

\begin{itemize}
    \item Wysoka kurtoza oznacza silne zgrupowanie ofert wśrod środka (1 pokoju).
    \item Z macierzy korelacji możemy zauważyc silne powiązanie tej zmiennej ze zmienna Accommodates - większość pokoi jest traktowana jako sypialnie.
\end{itemize}

\subsubsection{Availability 365}\label{subsubsec:availability-365}
Zmienna objaśniająca w modelu.
Opisuje ile dni w roku własciciel jest gotów wynajmować daną nieruchomość.
Wartość 0 oznacza, że własciciel w tym momencie nie chce wynajmować mieszkania (zamrożenie ogłoszenia).
Ustawienie takiej wartości oznacza, że nie można wynająć danej nieruchomości, ale ogloszenie wciąż jest dostępne.
Może to się wiązać z przejściem użytkownika na inna forme inwestycji lub próbe poprawienia jakości mieszkania, przed ponownym wystawieniem oferty.

\customtable{Availability365}
\customboxplot{Availability365}

Wnioski

\begin{itemize}
    \item Ponad połowa ogloszeń jest w stanie zamrożonym.
    Wynika to najprawdopodobniej z okresu zebrania danych (listopad) i może wiązać się np z chęcią podwyższenia cen na świeta, albo ogólnym brakiem zainteresowania wynajmem w tym okresie.
    \item Zmienna posiada bardzo duża wariancje.
    Wynika to z róznych rodzajów ofert na serwisie.
    Niektore są wystawiane przez duże firmy, a niektóre przez małych inwestorów (np ktoś wynajmuje część swojego domu).
\end{itemize}

\subsubsection{Accommodates}\label{subsubsec:accommodates}
Zmienna objaśniająca w modelu.
Określa dla ilu osób jest przystosowane mieszkanie (maksymalna liczba).

\customtable{Accommodates}

\customboxplot{Accommodates}

Wnioski

\begin{itemize}
    \item Większość mieszkań na wynajem jest przystosowana dla 2 osób.
    Zdaniem Deborah L. Bretta\cite{realestate-market-analysis} tego typu mieszkania są uznawane przez inwestorów za najopłacalniejsze.
    \item Skrajne obserwacje na wykresie dotyczą najprawdopodobniej wynajmowania przez własciciela całego mieszkania.
\end{itemize}

\subsubsection{Distance from Center}\label{subsubsec:distance-from-center}
Zmienna objaśniająca w modelu.
Określa odleglość rozważanej nieruchomości od centrum Berlina (52.5027778, 13.404166666666667) w kilometrach.
Zostało obliczone na podstawie wartości Latitude i Longtitude znajdujących sie w zbiorze.

\customplot{DistanceFromCenterPlot}{Rozkład ofert względem centrum miasta}{rozkład-polożenia}
Jak widać z powyższego rysunku, oferty są w równomiernie rozłożone wzgledem centrum.
Jednak większość najdroższych apartamentów znajduję sie w miare niedaleko od tego punktu.
\customtable{DistanceFromCenter}

\customboxplot{DistanceFromCenter}

Wnioski

\begin{itemize}
    \item Niski współczynnik skośności potwierdza, że oferty są w miare równo rozłożone wzgledem centrum.
    \item Od najdalej położonej nieruchomości na drugi koniec miasta trzeba pokonać prawie 25 km.
\end{itemize}

\subsubsection{Room Type}\label{subsubsec:room-type}
Zmienna objaśniająca w modelu.
Określa jaki rodzaj pokoju jest uwzględniony w ofercie.

\custompiechart{RoomType}

Wnioski

\begin{itemize}
    \item Ponad połowa ofert dotyczy wynajmu pojedynczych pokoi.
    Wynika to najprawdopodobniej z tego, że tego typu wynajem jest bardziej dochodowy przy duże ilości pokoi.
    \item Wspołdzielone pokoje sa mało popularne.
    Wiaże się to najprawdopodobniej z utrudnieniami w przewidzeniu współlokatorów, a co za tym idzie niską popularnościa tego typu ogloszeń.
\end{itemize}
