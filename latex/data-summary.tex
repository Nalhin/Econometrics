\section{Opis danych}\label{sec:opis-danych}

\subsection{Pochodzenie}\label{subsec:pochodzenie}
Dane pochodzą z serwisu Keggle\cite{berlin-airbnb-data}.
Zostały one zebrane na podstawie ofert wystawionych na serwisie Airbnb dnia 7 listopada 2018 roku.
Pozyskiwanie danych do zbioru odbyło się za pomocom technologi do webscrappingu.
Proces ten polega na pobraniu kodu HTML strony internetowej i przeparsowanie odpowiednich znaczników do postaci pliku csv.
Z uwagi na specyfikacje serwisu Airbnb, niektóre oferty mogą zawierać nieprawdziwe dane.
Skrajne i typowo bezsensowne wartości zostaną usunięte i nie beda prane pod uwagę przy dalszej analizie.

\subsection{Czyszczenie danych - Reguła Trzech Sigm}\label{subsec:czyszczenie-danych-reguła-trzech-sigm}

Z uwagi na specyfikacje pozyskania danych (występowanie błedów ludzkich) zdecydowałem sie na wykorzystanie reguły trzech sigm podczas czyszczenia danych.
Reguła trzech sigm dla danego rozkładu normalnego \(\mu , \sigma\) oznacza, że w przedziale \([\mu - 3\sigma ,\mu + 3\sigma ] \) znajduję się 99.7 \% wszystkich obserwacji.
Odstające (okolo 1500) obserwacje zostały usunięte z dalszej analizy.

\subsection{Problem braku danych}\label{subsec:problem-braku-danych}
W modelu znajdowała się niewielka ilość braków (okolo 50). Z uwagi na duża liczbe badanych obserwacji, braki zostały wyrzucone z dalszej analizy.

\subsection{Zmienne}\label{subsec:zmienne}

\customplot{CorrMatrix}{Macierz korelacji zmiennych}{corr-matrix}

\subsubsection{Price}\label{subsubsec:price}
Zmienna objaśniana w modelu.
Jest to cena za wynajem (w dolarach) na okres jednej doby.
Zmienne z wartościami zerowymi i abstrakcyjnie dużymi wartościami (siegające nawet 900\$) zostały uznane za outliery, a następnie usunięte z modelu.
Jako górna granice uznałem 500 \$.
W ten sposób z modelu zostały usunięte 82 obserwacje.


\customtable{Price}
\customboxplot{Price}

Wnioski

\begin{itemize}
    \item Z uwagi na wysoką dodatnią wartość skośności, jak rożnież różnic miedzy średnia, a mediana, można zauważyć, ze wykres jest mocno przsunięty w prawo, a zatem zawiera skrajnie wysokie wartości
    \item Wysoka kurtoza może oznaczać występowanie dodatnich outlierow (widoczne na wykresie), jednak z uwagi na specyfikacje zmiennej (na Airbnb można wynajmować nawet apartamenty), takowe nie bedą usuwane.
    \item Zmienna ma wariacje na wysokim poziomie, może to wynikać z dużych róznic ceny mieszkań zależnie od ich wielkości (i ilości osób mogących się w takowym zmieścić).
\end{itemize}

\subsubsection{Number of Reviews}\label{subsubsec:number-of-reviews}
Zmienna objaśniająca w modelu.

\customtable{NumberOfReviews}

\customboxplot{NumberOfReviews}

Wnioski

\begin{itemize}
    \item
\end{itemize}

\subsubsection{Minimum nights}\label{subsubsec:minimum-nights}
Zmienna objaśniająca w modelu.

\customtable{MinimumNights}

\customboxplot{MinimumNights}

Wnioski

\begin{itemize}
    \item
\end{itemize}

\subsubsection{Bathrooms}\label{subsubsec:bathrooms}
Zmienna objaśniająca w modelu.

\customtable{Bathrooms}

\customboxplot{Bathrooms}

Wnioski

\begin{itemize}
    \item
\end{itemize}

\subsubsection{Bedrooms}\label{subsubsec:bedrooms}
Zmienna objaśniająca w modelu.

\customtable{Bedrooms}

\customboxplot{Bedrooms}

Wnioski

\begin{itemize}
    \item
\end{itemize}

\subsubsection{Availability 365}\label{subsubsec:availability-365}
Zmienna objaśniająca w modelu.

\customtable{Availability365}
\customboxplot{Availability365}

Wnioski

\begin{itemize}
    \item
\end{itemize}

\subsubsection{Accommodates}\label{subsubsec:accommodates}
Zmienna objaśniająca w modelu.

\customtable{Accommodates}

\customboxplot{Accommodates}

Wnioski

\begin{itemize}
    \item
\end{itemize}

\subsubsection{Distance from Center}\label{subsubsec:distance-from-center}
Zmienna objaśniająca w modelu.

\customplot{DistanceFromCenterPlot}{Rozkład ofert względem centrum miasta}{rozkład-polożenia}

\customtable{DistanceFromCenter}

\customboxplot{DistanceFromCenter}

Wnioski

\begin{itemize}
    \item
\end{itemize}

\subsubsection{Room Type}\label{subsubsec:room-type}
Zmienna objaśniająca w modelu.

\custompiechart{RoomType}