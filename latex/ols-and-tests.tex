\section{Metoda najmniejszych kwadratów (MNK)}\label{sec:mnk}

\subsection{Opis}\label{subsec:opis}
Metoda najmniejszych kwadratów służy do estymacji parametrów w modelu ekonometrycznym.

\[ \alpha = (X^{T}X)^{-1}X^{T}y \]

\subsection{Założenia MNK}\label{subsec:założenia-mnk}

\begin{itemize}
    \item Zmienne objaśniające są nielosowe i nieskorelowane ze składnikiem losowym
    \item \[E(\epsilon) = 0\]
    \item \[Var(\epsilon) = \sigma^2 \iota\]
    \item \[rank(X) = (k+1) <n\]
\end{itemize}

\subsection{Twierdzenie Gaussa-Markowa}\label{subsec:twierdzenie-gaussa-markowa}
Przy spełnieniu założeń MNK estymator jest BLUE
\begin{itemize}
    \item \textbf{B}est
    \item \textbf{L}inear
    \item \textbf{U}nbiased
    \item \textbf{E}stimator
\end{itemize}

\section {Poprawnośc statystyczna modelu}\label{sec:testy-statystyczne-modelu}

\subsection{Koincydentność}\label{subsec:koincydentność}

\subsection{Występowanie efektu katalizy}\label{subsec:występowanie-efektu-katalizy}

Mówimy, że w modelu ekonometrycznym określonym przez regularną
parę korelacyjną (R, \(R_0\)) występuje efekt katalizy, jeżeli istnieje taka
para wskaźników (i, j), dla której

\[ r_{ij} < 0 \; lub \; r_{ij}  > \frac{r_i}{r_j} \]

Nateżenie efektu katalizy mierzymy wzorem

\[ \eta = R^2 - H \]

Gdzie H oznacza integralną pojemność informacyjną zmiennych objaśniających

\subsection{Istotność współczynnika determinacji}\label{subsec:istotność-współczynnika-determinacji}

\subsection{Istotność statystyczna pojedyńczych zmiennych objaśniających}\label{subsec:istotność-statystyczna-zmiennych-objaśniających}
Jeżeli spełnione jest założenie o normalności reszt, to możemy postawić następujące hipotezę

\begin{equation}
    \begin{split}
        H_0: \alpha_j = 0 \\
        H_1: \alpha_j \ne 0
    \end{split}
\end{equation}

Statystyka t ma postać \(t = \frac{\alpha j}{S_{\alpha j}}\).
Jeżeli \(|t| > t_{n -(k+1)}\) to odrzucamy \(H_0\).
Oznacza to brak wplywu zmiennej \(X_j\) na Y.
