\section{Wybór modelu}\label{sec:wybrany-modelu}

\subsubsection{Podział danych}

Dane zostały podzielone w następujący sposób:

\begin{itemize}
    \item 90 \% treningowe
    \item 10 \% testowe
\end{itemize}

Przy wykorzystaniu gotowego rozwiązania z biblioteki Pythonowej.

\subsubsection{Sposób wyboru}

Podczas wyboru modelu kierowałem się nastepującymi czynnikami

\begin{itemize}
    \item Maksymalizacja \(R^{2}\)
    \item Minimalizacja błędów ex-ante
    \item Poprawność statystyczna
\end{itemize}

Wybór modelu odbył się poprzez analize każdego możliwego wariantu doboru zmiennych.
Niestety mimo przeanalizwania wszystkich wariantów (i ich ewentualnych transformacji) nie udało znaleźć się modelu, który pomyślnie przeszedł wszystkie testy statystyczne.
Postać modelu jest już po przekształceniu na model homoskedastyczny.


\section{Postać wybranego modelu}\label{sec:postac-modelu}
Jest to postać modelu po transformacji celem zlikwidowania heteroskedastyczności.
W przypadku próby zamiany postaci modelu na np kwadratowy resultaty testów pogarszały się.

\begin{equation}
    \begin{split}
        y=-0.45624964184649164const \\ + 53.00011424464533Bedrooms \\ - 18.703573117550555RoomTypeShared room
    \end{split}
\end{equation}

\subsection{Wspolczynnik determinacji}\label{subsec:wspolczynnik-determinacji}

Wspolczynnik determinacji w modelu wynosi \(R^{2} = 0.9999962836185184\).

\subsection{Poprawność statystyczna modelu}\label{poprawnośc-statystyczna-modelu}

Model przeszedl poprawnie 9 na 11 testów.

\subsubsection{Efekt katalizy}\label{efekt-katalizy}


Brak zjawiska katalizy

\subsubsection{Istność statystyczna parametrow}\label{istotność-parametrów}

const
\begin{equation}
    \begin{split}
        &p-value= 1.4394961682323053e-57 \\ &t = -16.04704840214771  \\
    \end{split}
\end{equation}

RoomTypeSharedRoom
\begin{equation}
    \begin{split}
        &p-value= 2.222383942183182e-65  \\ &t = -17.142757081665643 \\
    \end{split}
\end{equation}


Bedrooms
\begin{equation}
    \begin{split}
         &p-value= 0.0  \\ &t = 71312.96528547468 \\
    \end{split}
\end{equation}


\subsubsection{Koincydencja}\label{koincydencja}

Model jest koincydentny


\begin{equation}
    \begin{split}
        &\alpha_1= 53.00011424464533  \\ &r_1 = 0.9999981129146659  \\
        &\alpha_2= -18.703573117550555  \\ &r_2 = -0.0025894676082915645  \\
    \end{split}
\end{equation}


\subsubsection{Istotność statystyczna współczynnika determinacji}\label{istotnosc-statystyczna-wspolczynnika-determinacji}

\begin{equation}
    \begin{split}
        &F= 2542786559.2096515 \\  &p-value = 0.0
    \end{split}
\end{equation}

Współczynnik determinacji jest istotny.

\subsubsection{Normalność reszt}\label{normalnosc-reszt}

\begin{equation}
    \begin{split}
        &JB=436502107.97198707 \\ &p-value = 0.0
    \end{split}
\end{equation}

Test Jarque-Bera nie potwierdza normalności reszt.
Niezaleznie od wyniku tego testu można powołac sie na twierdzenie graniczne.


\subsubsection{Stabilnosc postaci analitycznej}\label{stabilnosc-postaci-analitycznej}

Test Ramseya RESET
\begin{equation}
    \begin{split}
        &F = 7546.829969968404 \\ &p-value = 0.0
    \end{split}
\end{equation}

Postać modelu nie jest stabilna.

\subsubsection{Liniowość modelu}\label{liniowosc-modelu}
Test liczby serii
\begin{equation}
    \begin{split}
        &Z = -134.27980839321282 \\ &p-value = 0.0
    \end{split}
\end{equation}

Postać modelu nie jest liniowa.


\subsubsection{Współliniowość}\label{współliniowość}

W modelu nie występuje współliniowość.

\subsubsection{Autokorelacja składnika losowego}\label{autokorelacja-składnika-losowego}
Test Breuscha-Godfreya
\begin{equation}
    \begin{split}
        &\chi^{2}= 0.05347663492978216 \\ &p-value =0.8171204664638051
    \end{split}
\end{equation}

Brak autokorelacji I rzędu.

\subsubsection{Homoskedastyczność}
Test Breuscha-Pagana
\begin{equation}
    \begin{split}
        &LM=0.15303357608376977 \\  &p-value = 0.926337362767837
    \end{split}
\end{equation}

Reszty są homoskedastyczne.

\newpage

\section{Prognoza}\label{sec:prognoza}

Błędy prognozy modelu prezentują wyglądają następująco (dla danych testowych).

\begin{equation}
    \begin{split}
        &ME = -0.6606837119514631 \\
        &MAE = 28.85512409962552 \\
        &RMSE = 42.97309038468132
    \end{split}
\end{equation}

Jak widać model nie jest aż tak skuteczny dla danych testowych jak wskazuje na to współczynnik determinacji.
