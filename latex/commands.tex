\newcommand{\customboxplot}[1]{
    \begin{figure}[H]
        \center
        \import{generated/}{#1BoxPlot.pgf}
        \caption{Wykres pudelkowy zmiennej \textbf{#1}}
        \label{fig:box-plot-#1}
    \end{figure}
}

\newcommand{\custompiechart}[1]{
    \begin{figure}[H]
        \import{generated/}{#1PieChart.pgf}
        \caption{Diagram kołowy zmiennej \textbf{#1}}
        \label{fig:pie-chart-#1}
    \end{figure}
}

\newcommand{\customplot}[3]{
    \begin{figure}[H]
        \import{generated/}{#1.pgf}
        \caption{#2}
        \label{fig:plot-#3}
    \end{figure}
}

\newcommand{\customtable}[1]{
    \import{generated/}{#1Table.tex}
}

\newcommand{\variabletable}[2]{
    \begin{table}[H]
        \centering
        \begin{tabular}{ |p{3cm}|p{3cm}|}
            \hline
            \multicolumn{2}{|c|}{#1} \\
            \hline
            #2
            \hline
        \end{tabular}
        \caption{Statystyki opisowe zmiennej #1}
        \label{tab:summary-#1}
    \end{table}
}
