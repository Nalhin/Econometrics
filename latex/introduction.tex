\section{Opis}\label{sec:opis}

Przed pandemią koronawirusa wynajem krótkoterminowy cieszył się olbrzymią popularnościa szczególnie w Krakowie.
Sukcesy serwisów takich jak Airbnb (amerykańska firma założona w sierpniu 2008), który w ciągu ostatnich 12 lat zdominował ten rynek.
Polscy inwestorzy uważają mieszkania za jedną z najlepszych opcji ulokowania pieniędzy.
Cechuje się ona bardzo niskim poziomem ryzyka i dość wysoką stopą zwrotu (okolo 7\% w skali roku przy wyjamnie długoterminowym).
Wysoka popularność tego typu inwestycji sprawiła, że w ciągu ostatniego roku w Krakowie mieszkania podrożaly o prawie 15\%.
Mimo swojej dużej stabilności cieżko powiedzieć jak ten rynek poradzi sobie z pandemią.
Podniesienie wkładu własnego podczas zawierania kredytu hipotecznego z 10\% do 30\% oraz zmniejszenie liczby osób chetnych na wynajem może dość mocno ostudzić ten rynek

\section{Cel}\label{sec:cel}
Celem projektu jest zbadanie czynników wpływających na ceny wynajmu krótkoterminowego (za jedną noc).
Analiza taka pozwoli zbadać jakie mieszkania są szczególnie atrakcyjne dla potencialnych wynajmujących i czym należy się kierować przy wyborze mieszkania dla tego typu inwestycji.

\section{Biblioteki}\label{sec:biblioteki}

W moim projekcie zostały wykorzystane następujące biblioteki
\begin{itemize}
    \item Python
    \begin{itemize}
        \item Pandas
        \item Numpy
        \item Matplotlib
        \item Seaborn
        \item Scipy
    \end{itemize}
\end{itemize}

Większość kodu i testów została napisana własnoręcznie (bez używania gotowych bibliotek), a proces wyboru zmiennych w dużej mierze zautomatyzowany.